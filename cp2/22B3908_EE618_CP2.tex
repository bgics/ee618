\documentclass[11pt,a4paper]{article}
\usepackage[margin=1in]{geometry}
\usepackage{graphicx}
\usepackage{amsmath}
\usepackage{float}
\usepackage{booktabs}
\usepackage{multirow}
\usepackage{tabularx}
\usepackage{array}
\usepackage{caption}
\usepackage{subcaption} 

% \setcounter{secnumdepth}{0}

\begin{document}

\begin{titlepage}
	\centering

	\huge \textbf{EE 618: CMOS Analog VLSI Design} \\
	\vspace{1cm}

	\LARGE \textsc{Course Project 2} \\
	\vspace{5cm}

	\includegraphics[width=0.55\textwidth]{../icon.png} \\
	\vspace{5cm}

	\Large
	\begin{tabular}{rl}
		\textbf{Student Name:}    & Bhuvansh Goyal    \\
		\textbf{Roll Number:}     & 22B3908           \\
		\textbf{Instructor:}      & Prof. Rajesh Zele \\
		\textbf{Submission Date:} & 17 Nov 2025
	\end{tabular}

	\vfill
\end{titlepage}

\clearpage

\tableofcontents

\clearpage

\section{RC Compensation}

\subsection{DRC and LVS Checks}

The layout was verified using DRC and LVS. One warning appeared during extraction, but as confirmed by the TAs, it is safe to ignore. The corresponding screenshots are provided below.

\begin{figure}[H]
	\centering

	\begin{subfigure}{0.45\textwidth}
		\centering
		\includegraphics[width=\textwidth]{images/rc_drc.png}
		\caption{DRC check}
	\end{subfigure}
	\hfill
	\begin{subfigure}{0.45\textwidth}
		\centering
		\includegraphics[width=\textwidth]{images/lvs_1.png}
		\caption{LVS report (1)}
	\end{subfigure}

	\vspace{0.5cm}

	\begin{subfigure}{0.45\textwidth}
		\centering
		\includegraphics[width=\textwidth]{images/lvs_2.png}
		\caption{LVS report (2)}
	\end{subfigure}
	\hfill
	\begin{subfigure}{0.45\textwidth}
		\centering
		\includegraphics[width=\textwidth]{images/lvs_3.png}
		\caption{LVS report (3)}
	\end{subfigure}

	\caption{Summary of DRC and LVS verification results. The extraction warning shown is approved to be ignored.}
\end{figure}

\subsection{PEX Extraction}

\begin{figure}[H]
	\centering
	\includegraphics[width=0.5\textwidth]{images/extraction_PEX.png}
	\caption{Successful PEX extraction summary from Quantus.}
\end{figure}

\clearpage

\subsection{Layout}

The final layout occupies an area of \textbf{22.5\,µm × 20.4\,µm}, as shown in Fig.~\ref{fig:ota_layout}.

\begin{figure}[H]
	\centering
	\includegraphics[width=\textwidth]{images/rc_ota_layout.png}
	\caption{Final OTA layout showing device placement and routing.}
	\label{fig:ota_layout}
\end{figure}

Common-centroid techniques were used to ensure accurate matching across all critical device groups.
The current-mirror transistors \textbf{(NM6, NM7, NM9)} were implemented in a common-centroid arrangement with identical unit device dimensions.
The differential pair \textbf{(NM12, NM13)} was also laid out using a common-centroid pattern to suppress gradient-induced mismatch.

The PMOS devices \textbf{(PM13, PM14, PM15)} were similarly placed in a common-centroid configuration to maintain uniform device characteristics and reduce systematic mismatch.

The overall layout was made fully symmetric wherever possible, further minimizing mismatch due to process gradients and ensuring consistent electrical behavior across mirrored structures.

All matched devices use identical unit-cell geometry and segmentation, providing symmetry and improved layout-level matching.
\subsection{DC Operating Point}

\subsubsection{Component Parameters}
\begin{figure}[H]
	\centering
	\includegraphics[width=\textwidth]{images/rc_component_parameters_PEX.png}
	\caption{Component parameters.}
\end{figure}

\subsubsection{TT (Schematic)}
\begin{figure}[H]
	\centering
	\includegraphics[width=\textwidth]{../cp1/images/rc_dc_op_TT.png}
	\caption{DC operating points — TT.}
\end{figure}

\begin{figure}[H]
	\centering
	\includegraphics[width=0.4\textwidth]{../cp1/images/rc_pwr_TT.png}
	\caption{Supply current measurement — TT.}
\end{figure}

\subsubsection{TT (PEX)}
\begin{figure}[H]
	\centering
	\includegraphics[width=\textwidth]{images/rc_dc_op_PEX.png}
	\caption{DC operating points — TT.}
\end{figure}

In PEX simulations the currents show up divided by the MOSFET multiplier, unlike the schematic results.

\begin{figure}[H]
	\centering
	\includegraphics[width=0.4\textwidth]{images/rc_pwr_PEX.png}
	\caption{Supply current measurement — TT.}
\end{figure}

\subsection{Stability Analysis}

\subsubsection{TT (Schematic)}
\begin{figure}[H]
	\centering
	\includegraphics[width=\textwidth]{../cp1/images/rc_stab_bode_TT.png}
	\caption{Bode magnitude and phase plot — TT.}
\end{figure}

\begin{figure}[H]
	\centering
	\includegraphics[width=0.6\textwidth]{../cp1/images/rc_stab_summary_TT.png}
	\caption{Stability Summary — TT.}
\end{figure}

\subsubsection{TT (PEX)}
\begin{figure}[H]
	\centering
	\includegraphics[width=\textwidth]{images/rc_stab_TT_PEX.png}
	\caption{Bode magnitude and phase plot — TT.}
\end{figure}

\begin{figure}[H]
	\centering
	\includegraphics[width=0.6\textwidth]{images/rc_stab_TT_summary_PEX.png}
	\caption{Stability Summary — TT.}
\end{figure}

\subsubsection{Schematic vs PEX (TT)}
\begin{figure}[H]
	\centering
	\includegraphics[width=\textwidth]{images/rc_stab_combined.png}
	\caption{Bode magnitude and phase plot — Schematic vs PEX.}
\end{figure}

\subsection{AC Analysis: Differential Gain}

\subsubsection{TT (Schematic)}
\begin{figure}[H]
	\centering
	\includegraphics[width=\textwidth]{../cp1/images/rc_dm_bode_TT.png}
	\caption{Closed-loop gain and phase plot — TT.}
\end{figure}

\begin{figure}[H]
	\centering
	\includegraphics[width=0.4\textwidth]{../cp1/images/rc_offset_TT.png}
	\caption{Input-referred systematic offset — TT.}
\end{figure}

\begin{figure}[H]
	\centering
	\includegraphics[width=\textwidth]{../cp1/images/rc_dm_dc_op_TT.png}
	\caption{DC operating points — TT.}
\end{figure}

\subsubsection{TT (PEX)}
\begin{figure}[H]
	\centering
	\includegraphics[width=\textwidth]{images/rc_dm_TT_PEX.png}
	\caption{Closed-loop gain and phase plot — TT.}
\end{figure}

\begin{figure}[H]
	\centering
	\includegraphics[width=0.4\textwidth]{images/rc_dm_offset_PEX.png}
	\caption{Input-referred systematic offset — TT.}
\end{figure}

\begin{figure}[H]
	\centering
	\includegraphics[width=\textwidth]{images/rc_dm_op_PEX.png}
	\caption{DC operating points — TT.}
\end{figure}

In PEX simulations the currents show up divided by the MOSFET multiplier, unlike the schematic results.

\subsubsection{Schematic vs PEX (TT)}
\begin{figure}[H]
	\centering
	\includegraphics[width=\textwidth]{images/rc_dm_combined.png}
	\caption{Closed-loop gain and phase plot — Schematic vs PEX.}
\end{figure}

% \subsection{AC Analysis: Common Mode Gain}
%
\subsubsection{TT (Schematic)}
\begin{figure}[H]
	\centering
	\includegraphics[width=\textwidth]{../cp1/images/rc_cm_bode_TT.png}
	\caption{Open-loop common-mode gain plot — TT.}
\end{figure}

\subsubsection{TT (PEX)}
\begin{figure}[H]
	\centering
	\includegraphics[width=\textwidth]{images/rc_cm_PEX.png}
	\caption{Open-loop common-mode gain plot — TT.}
\end{figure}

\subsubsection{Schematic vs PEX (TT)}
\begin{figure}[H]
	\centering
	\includegraphics[width=\textwidth]{images/rc_cm_combined.png}
	\caption{Open-loop common-mode gain plot — Schematic vs PEX.}
\end{figure}

\subsection{Transient Analysis: Sinusoidal Input}

\subsubsection{TT (Schematic)}
\begin{figure}[H]
	\centering
	\includegraphics[width=\textwidth]{../cp1/images/rc_sin_TT.png}
	\caption{Input and output transient waveforms — TT.}
\end{figure}

\subsubsection{TT (PEX)}
\begin{figure}[H]
	\centering
	\includegraphics[width=\textwidth]{images/rc_sin_waveform_PEX.png}
	\caption{Input and output transient waveforms — TT.}
\end{figure}

\subsection{Transient Analysis: Step Input}

\subsubsection{TT (Schematic)}
\begin{figure}[H]
	\centering
	\includegraphics[width=\textwidth]{../cp1/images/rc_slew_TT.png}
	\caption{Output derivative plot — TT.}
\end{figure}

\begin{figure}[H]
	\centering
	\includegraphics[width=\textwidth]{../cp1/images/rc_ts_TT.png}
	\caption{Settling time plot (1\% accuracy) — TT.}
\end{figure}

\subsubsection{TT (PEX)}
\begin{figure}[H]
	\centering
	\includegraphics[width=\textwidth]{images/rc_slew_deriv_PEX.png}
	\caption{Output derivative plot — TT.}
\end{figure}

\begin{figure}[H]
	\centering
	\includegraphics[width=\textwidth]{images/rc_slew_PEX.png}
	\caption{Settling time plot (1\% accuracy) — TT.}
\end{figure}

\subsubsection{Schematic vs PEX (TT)}
\begin{figure}[H]
	\centering
	\includegraphics[width=\textwidth]{images/rc_slew_combined.png}
	\caption{Output derivative plot — Schematic vs PEX.}
\end{figure}

\subsection{Noise analysis}

\subsubsection{TT (Schematic)}
\begin{figure}[H]
	\centering
	\includegraphics[width=\textwidth]{../cp1/images/rc_noise_TT.png}
	\caption{Input-referred noise PSD — TT.}
\end{figure}

\begin{figure}[H]
	\centering
	\includegraphics[width=0.5\textwidth]{../cp1/images/rc_noise_summary_TT.png}
	\caption{Noise summary — TT.}
\end{figure}

\subsubsection{TT (PEX)}
\begin{figure}[H]
	\centering
	\includegraphics[width=\textwidth]{images/rc_noise_PEX.png}
	\caption{Input-referred noise PSD — TT.}
\end{figure}

\begin{figure}[H]
	\centering
	\includegraphics[width=0.5\textwidth]{images/rc_noise_summary_PEX.png}
	\caption{Noise summary — TT.}
\end{figure}

\subsubsection{Schematic vs PEX (TT)}
\begin{figure}[H]
	\centering
	\includegraphics[width=\textwidth]{images/rc_noise_combined.png}
	\caption{Input-referred noise PSD — Schematic vs PEX.}
\end{figure}

\subsection{Summary of results obtained}

\begin{table}[H]
	\centering
	\renewcommand{\arraystretch}{1.2}
	\setlength{\tabcolsep}{5pt}
	\begin{tabularx}{0.9\textwidth}{|c|>{\raggedright\arraybackslash}X|c|c|}
		\hline
		\textbf{Q.No}      & \textbf{Parameters}                                        & \textbf{Layout Results} & \textbf{Schematic Results} \\
		\hline
		2                  & Power Consumption (mW)                                     & 0.261                   & 0.261                      \\
		\hline
		\multirow{4}{*}{3} & DC gain (dB)                                               & 54.922                  & 54.928                     \\ \cline{2-4}
		                   & f-3dB (kHz)                                                & 207.993                 & 215.725                    \\ \cline{2-4}
		                   & Unity Gain Frequency (MHz)                                 & 110.413                 & 118.459                    \\ \cline{2-4}
		                   & Phase margin ($^\circ$)                                    & 64.3908                 & 67.574                     \\
		\hline
		\multirow{3}{*}{4} & Closed Loop Gain (mdB)                                     & -6.256                  & -5.466                     \\ \cline{2-4}
		                   & f-3dB (MHz)                                                & 183.598                 & 195.636                    \\ \cline{2-4}
		                   & Input referred offset ($\mu$V)                             & 203.70                  & 123.70                     \\
		\hline
		\multirow{2}{*}{5} & Common-mode gain (dB)                                      & -4.004                  & -3.178                     \\ \cline{2-4}
		                   & CMRR (dB)                                                  & 58.926                  & 58.106                     \\
		\hline
		6                  & Output Swing (Vpp) (mV)                                    & 599.394                 & 599.655                    \\
		\hline
		\multirow{2}{*}{7} & Slew rate (V/$\mu$s)                                       & 101.672                 & 112.734                    \\ \cline{2-4}
		                   & Settling Time (1\% accuracy) (ns)                          & 3.48                    & 3.50                       \\
		\hline
		\multirow{4}{*}{8} & Input referred spot noise @100 kHz (nV/$\sqrt{\text{Hz}}$) & 59.677                  & 60.034                     \\ \cline{2-4}
		                   & Input referred spot noise @10 MHz (nV/$\sqrt{\text{Hz}}$)  & 17.396                  & 17.863                     \\ \cline{2-4}
		                   & Total summarised Noise ($\mu$V)                            & 285.557                 & 291.234                    \\ \cline{2-4}
		                   & Total input-referred Noise ($\mu$V)                        & 386.885                 & 382.986                    \\
		\hline
	\end{tabularx}
\end{table}

\subsection{Comments}

In the PEX simulations, the magnitude plot flattens out at high frequencies. This behaviour is caused by \textbf{parasitic zeros} introduced by layout effects such as feedthrough capacitances. These zeros contribute a \(+20\,\text{dB/dec}\) slope, which cancels the dominant pole’s \(-20\,\text{dB/dec}\) roll-off, leading to a flat high-frequency gain response.

The Unity-Gain Frequency (UGF) also decreases after PEX. Since
\( f_{\text{UGF}} \approx g_{m1} / (2\pi C_c) \), an increase in the effective compensation capacitance \(C_c\) due to parasitics directly lowers the UGF.

For the same reason, the Slew Rate (SR) degrades. Using the relation
\( \text{SR} = I_{SS}/C_c \), the increased \(C_c\) reduces the SR from
\(112\,\text{V}/\mu\text{s}\) in the schematic to \(101\,\text{V}/\mu\text{s}\) in the PEX results.

All other steady-state performance parameters remain consistent with the schematic behaviour.

\end{document}
